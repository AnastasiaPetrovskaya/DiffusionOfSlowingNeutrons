\documentclass[a4paper]{article}

\usepackage[english, russian]{babel}
\usepackage[utf8x]{inputenc}
\usepackage{amsmath}
\usepackage{pgfplots}
\usepackage{rotating}
\usepackage{graphicx}
\usepackage{indentfirst}
\usepackage[colorinlistoftodos]{todonotes}
\usepackage[left=2.5cm, top=2cm, right=1.5cm, bottom=2.5cm,
            nohead]{geometry}
\title{Диффузия замедляющихся нейтронов.}
\author{
    Напечатано по заказу кафедры K01-22М:
    \and
    Петровская А.В., Бодунков Д.В., Таракчян Л.С., Минаков А.О.}
    \date{}
\begin{document}
    \maketitle
    В предыдущих разделах рассмотрены процессы замедления
    и диффузии в отрыве друг от друга. При рассмотрении процесса
    замедления не учитывался факт пространственного перемещения,
    а при изучении диффузии нейтронов не учитывался их факт
    пространственного перемещения, а при изучении диффузии нейтронов
    пренебрегалось изменениями энергии нейтронов при рассеянии на
    ядрах среды. В действительности эти процессы происходят
    одновременно: нейтроны сталкиваясь с ядрами среды перемещаются
    в пространстве и изменяют свою энергию. Поэтому при вычислении
    распределения плотности потока нейтронов в ядерном реакторе
    нельзя разделять процессы замедления и диффузии. Наиболее
    простой математической моделью, позволяющей описать диффузию
    замедляющихся нейтронов является модель непрерывного замедления.

    Основное положение этой модели заключаетмя в том, что
    дискретный процесс потери энергии нейтроном, при замедлении
    аппроскимируется непрерывной зависимостью ( см. рис )

    \tikzset{
        every pin/.style={fill=yellow!50!white,rectangle,
                        rounded corners=3pt,font=\tiny},
        small dot/.style={fill=black,circle,scale=0.3}
    }
    \begin{tikzpicture}
        \begin{axis}[
            clip=false,
            xmin = 0.4,
            ymin = 0,
            ymax = 1.8,
            ylabel = {\text{\begin{turn}{270}$E$\end{turn}}},
            xlabel = {$t$},
            xtick=\empty, ytick=\empty,
            axis x line = bottom,
            axis y line = left,
            legend to name = legendname,
        ]

            \addplot[blue] {1 - log10(x)};
            \addplot+[const plot]{1 - log10(x)};
            \legend{
                приближение непрерывного замедления,
                реальное изменение энергии нейтронов в процессе замедления
            }
            \node[small dot,pin=120:{$E_0$}] at (yticklabel cs:0.77) {};
        \end{axis}
    \end{tikzpicture}
    \newline
    \ref{legendname}

    Найдем функциональную связь между временем и энергией
    при непрерывном торможении нейтрона. Пусть нейтрон при
    своем замедлени проходит энергетический интервал $dE$,
    около энергии $E$ за время $dt$. Нейтрон снижает свою
    энергию за счет того, что за время dt сталкивается с ядрами среды.

    Число таких столкновений при диффузии нейтрона легко определяется 
    из соотношений
    \begin{equation}
        \frac{V}{l_s}dt
        \text{, где $V$ - скорость нейтрона}
    \end{equation}
    соответствующая энергия $E$.

    С другой стороны, число столкновений, которое необходимо
    претерпеть нейтрону, чтобы изменить свою энергию на величину $dE$,
    есть отношение приращения логарифма энергии на этом интервале к
    величине $\xi$ - средней потере логарифма энергии на одно
    столкновение. Приравняем эти величины и ваполняя простые
    преобразования, получим:
    \begin{equation}
        \frac{dE}{dt} = -\frac{V}{l_s}\xi E = \xi \Sigma_s V E
    \end{equation}
    Знак ($-$) в этом выражении взят с целью описать факт уменьшения
    энергии нейтрона со временем.

    Обратимся теперь к следующей задаче: в бесконечной непоглощающей
    среде находится точечный источник, испускающий нейтроны с энергией
    $E_o$. Если источник испускает в единицу времени какую-то порцию
    нейтронов, то эти нейтроны будут распределяться по все
    возрастающему объему. Поэтому число нетронов в $I \text{ см}^3$
    около точки с координатой $\vec{r}$, будет зависеть от
    хронологического времени
    $t$, т.е. \( n_1 = n_1 (\vec{r},t) \).

    Изменение плотности нейтронов $n_1(\vec{r},t)$ при
    отсутствии поглощения происходит только за счет диффузии, поэтому:
    \begin{equation}
        \frac{\partial n_1}{\partial t} = D V_\Delta n_1
    \end{equation}

    Уравнение (3) описывает изменение плотности нейтронов, за счет
    того, что источник испустил порцию нейтронов, равную мощности
    источники, то есть, по сути дела, уравнения (3) описывает скорость
    изменения числа нейтронов, т.е. 
    \(n_1(\vec{r}, t)=\frac{dn}{dt}\).
    Учтем, что переменные $t$ и $E$ связанны соотношением (1).
    Поскольку форма дифференциала $dn$ не зависит от того, что
    рассматривать в качемтве переменной, имеем
    \begin{equation}
        dn = \frac{dn}{dt}dt \text{\;\;\;\;\;\;или\;\;\;\;\;\;}
        dn = \frac{dn}{dE}dE
    \end{equation}
    откуда
    \begin{equation}
        \frac{dn}{dt}dt = \frac{dn}{dE}dE
    \end{equation}
    Обозначим \(\frac{dn}{dE}=n_2(\vec{r},E)\), тогда
    будем иметь
    \begin{equation}
        n_1(\vec{r},t)dt = n_2(\vec{r},E)dE
    \end{equation}
    откуда
    \begin{equation}
        n_1(\vec{r},t) =
        n_2(\vec{r},E)\frac{dE}{dt} =
        n_2(\vec{r},E) V \xi E \Sigma_s
    \end{equation}
    или
    \begin{equation}
        n_1(\vec{r},t) =
        \varPhi(\vec{r}, E) \xi \Sigma_s E
    \end{equation}

    $\frac{dn}{dE}$ - есть число нейтронов в ед. объема,
    приходящихся на единичный энергетический интервал, т.е.
    $\varphi(\vec{r},E)$.

    Величина \(q(\vec{r},\vec{E})=
    \xi\Sigma_s E\varphi(\vec{r},E) \)
    носит название плотности замедления и имеет смысл числа нейтронов
    в $I\text{ см}^3$ пересекающих в ед. времени значение энергии $E$.

    Действительно, величина $\xi$ есть среднее изменение логорифма
    энергии в одном акте рассеяния
    \begin{equation}
        \xi = \overline{_\Delta ln E} \approx
        \overline{(ln E)^{'}_E {}_\Delta E} =
        \overline{\frac{1}{E} {}_\Delta E}
        \text{, откуда } \overline{{}_\Delta E} = \xi E
    \end{equation}

    $\overline{{}_\Delta E}$ - потеря энергии нейтроном в одном акте
    рассеяния. Если интервал $\overline{{}_\Delta E}$ расположен
    между $E$ и $E + \overline{{}_\Delta E}$, то каждое рассеяние
    приводит к снижению энергии нейтрона за значение $E$.

    Число нейтронов претерпевших рассеяние в интервале
    $[E, E + \overline{{}_\Delta E}]$, есть произведение числа
    нейтронов рассеяных в единичном интервале энергий
    $\varphi(E)\Sigma_s$ на величину
    $\overline{{}_\Delta E}$. Все эти нейтроны снижают свою энергию
    за значение $E$, следовательно
    \begin{equation}
        q(\vec{r},E) =
        \varphi(E)\Sigma_s\overline{{}_\Delta E} =
        \Sigma_s\varphi(E)E
    \end{equation}

    Так как \(\frac{\partial n_1}{\partial E} =
    \frac{\partial n_1}{\partial E} \frac{\partial E}{\partial t}\)
    , и \(n_1 = q(\vec{r},E)\), получим из уравнения (3)
    относительно плотности нейтронов, уравнение (23) относительно
    плотности замедления
    \begin{equation}
        D_\Delta q(\vec{r}, E) = -\xi E\Sigma_s
        \frac{\partial q(\vec{r}, E)}{\partial E}
    \end{equation}
    Уравнение (23) можно еще упростить, если ввести новую независимую
    переменную
    \begin{equation}\nonumber
        \tau(E) = \int_E^{E_0} \frac{D}{\Sigma_s}\frac{dE}{\xi E}
    \end{equation}
    \begin{equation}
        \text{Очевидно, что }
        \frac{\partial q}{\partial \tau}
        \frac{\partial \tau}{\partial E} =
        \frac{\partial q}{\partial E}
        \text{; откуда}
    \end{equation}
    \begin{equation}
        \frac{\partial q}{\partial \tau} =
        \frac{\partial q}{\partial E}
        \frac{1}{\frac{\partial \tau}{\partial E}}
        \text{ , но }
        \frac{\partial \tau}{\partial E} =
        -\frac{D}{\Sigma_s\xi E}
    \end{equation}
    \begin{equation}\nonumber
        \text{т.е. }
        \frac{\partial q}{\partial \tau} =
        -\frac{\xi\Sigma_s E}{D}\frac{\partial q}{\partial E}
    \end{equation}
    Тогда уравнение (23) запишется в следующем виде
    \begin{equation}\label{e_24}\tag{24}
        {}_\Delta q(\vec{r}, \tau) =
        \frac{\partial q(\vec{r}, \tau)}{\partial \tau}
    \end{equation}
    Уравнение (\ref{e_24}) описывает распределение в пространстве
    $\vec{r}$ и в пространстве $\tau$ плотности
    замедления $q(\vec{r}, \tau)$. Величина $\tau$ - носит
    специальное название - "возраст нейтрона" и имеет размерность
    $[\text{см}^2]$. Само уравнение (\ref{e_24}) является уравнением
    теплопроводности. Решение этого уравнения для точечного источника
    в бесконечной среде имеет вид:
    \begin{equation}\label{e_25}\tag{25}
        q(\vec{r},\tau) =
        \frac{Q}{(4\pi\tau)^{3/2}}
        \exp{(-\frac{r^3}{4\tau})}
    \end{equation}
    На рис. 9 показан качественный вид решения (\ref{e_25}) в
    зависимости от координаты при различных значениях параметра $\tau$.

    \tikzset{
        every pin/.style={fill=yellow!50!white,rectangle,
                        rounded corners=3pt},
        small dot/.style={fill=black,circle,scale=0.3}
    }
    \begin{tikzpicture}
        \begin{axis}[
            clip=false,
            axis x line = bottom,
            axis y line = center,
            xtick=\empty, ytick=\empty,
            xmin = -6, xmax = 6,
            ymin = 0,
            ymax = 10,
            ylabel = {$q$},
            xlabel = {$r$},
        ]

            \addlegendimage{empty legend}
            \addlegendentry{
                $\tau_2 > \tau_1$
            }
            \addplot[blue] {4*cos(deg(x/2))+4};
            \addplot[red]{2*cos(deg(x/2))+4};
            \node[small dot,pin=120:{$\tau_1$}] at (yticklabel cs:0.8) {};
            \node[small dot,pin=250:{$\tau_2$}] at (yticklabel cs:0.6) {};
        \end{axis}
    \end{tikzpicture}

    Если $\tau$ мало, то это означает, что энергия нейтронов
    достаточно близка к энергии нейтронов источника $E_0$ и кривая
    $q(r,\tau)$ становится более выровненным. Важным случаем является
    тот, когда
    \begin{equation}\nonumber
        \tau=\int_{E_t}^{E_0}\frac{D}{\xi\Sigma_s}\frac{dE}{E}=
        \tau_\tau
    \end{equation}
    где $E_T$ - энергия тепловых нейтронов. В этом случае 
    $q(r, \tau_\tau)$ - дает распределение источников тепловых
    нейтронов около точечного источника быстрых нейтронов. Физический
    смысл понятия возраста нейтронов $\tau$ заключается в том, что
    возраст нейтронов $\tau(E)$ есть величина пропорциональная среднему
    квадрату смещения нейтронов от точки их рождения до точки, где
    их энергия равна величине $E$. Действительно, средний квадрат
    смещения нейтрона до достижения возраста $\tau$ есть
    \begin{equation}
        \vec{r}_{\tau}^2 =
        frac{\int_0^{\infty}r^2 q(r, \tau)4\pi r^2 dr}
        {\int_0^{\infty}q(r,\tau)\pi r^2 dr} =
        \frac{1}{Q}\frac{Q4\pi}{{4\pi\tau}^{3/2}}
        \int_0^\infty r^4 e^{-\frac{r^2}{t\tau}}dr = 6\tau
    \end{equation}
    При получении этого результата \(\vec{r}^2 = 6\tau\)
    учтено, что \(\int_0^\infty q(r,\tau)4\pi r^2 dr\), т.е. число
    нейтронов замедляющихся до возраста $\tau$ в ед. времени во всем
    объеме рассматриваемой среды равно мощности источника $Q$.
    Поскольку возраст нейтронов $\tau(E_T)$ пропорционален смещению
    нейтрона от точки рождения (в качестве быстрого нейтрона) до
    точки замедления до тепловой энергии $E_T$, а квадрат длины
    диффузии $L^2$ пропорционален смещению от точки рождения
    теплового нейтрона до точки поглощения, то величина 
    \(M^2 = \tau + L^2\) - пропорциональна среднему смещению нейтрона
    от точки его рождения как быстрого нейтрона до точки его
    поглощения как теплового нейтрона ( см.рис. 10). Величина
    $M^2$ называется площадью миграции нейтрона.

    \setlength{\unitlength}{1cm}
    \begin{picture}(7,6)
            \put(1,1){\line(5,1){5}}
            \put(6,2){\line(-1,3){1}}
            \put(1,1){\vector(1,1){4}}
            \put(0.6,0.6){$1$}
            \put(6.3,1.7){$2$}
            \put(5.3,5.3){$3$}
            \put(3.6,1){$6\tau$}
            \put(6,3.3){$6L^2$}
            \put(2.3,3.8){$6M^2$}
    \end{picture}

    1 - точка, где родился быстрый нейтрон
    \newline
    2 - точка, где быстрый нейтрон, замедлился до тепловой
    энергии и стал тепловым (точка рождения теплового нейтрона)
    \newline
    3 - точка поглощения теплового нейтрона

    Важными характеристиками являются время диффузии и время
    замедления нейтронов до тепловой энергии. При нормальных условаиях
    в качестве тепловой энергии нейтрона $E_T$ принимается величина
    \(E_T=0,025\text{эВ}\), что соответствует скорости нейтронов
    \(V_T=2200\frac{\text{м}}{\text{сек}}\).

    Среднее время диффузии нейтрона до поглощения определяется из
    выражения
    \begin{equation}\label{e_27}\tag{27}
        t_T=\frac{l_U}{V_T}=\frac{1}{\Sigma_u V_T}
    \end{equation}
    $\Sigma_u$ - макроскопическое сечение поглощения среды

    Среднее время замедления нейтрона от энергии $E_0$ до энергии
    $E_T$ определяется с помощью выражения (20)
    \begin{eqnarray}\nonumber
        t_\text{зам} = \int_0^{t_\text{зам}} dt =
        \int_{E_T}^{E_0} - \frac{dE}{\xi \Sigma_s VE} =
        \int_{E_T}^{E_0} - \frac{dE}{\xi\Sigma\sqrt{2}E^{3/2}} =
        \\ \nonumber
        \frac{1}{\sqrt{2}\xi\Sigma_s}
        \left[-\frac{1}{\left(\frac{3}{2}-1\right)E^{3/2-1}}\right]
        \Bigg|_{E_T}^{E_0} = \frac{1}{\sqrt{2}\xi\Sigma_s\frac{1}{2}}
        \left(-\frac{1}{\sqrt{E}}\right)\Bigg|_{E_T}^{E_0} =
        \\ \nonumber
        \frac{2}{\xi\Sigma_s}\left(-\frac{1}{V}\right)
        \Bigg|_{V_T}^{V_0}=\frac{2}{\xi\Sigma_s}
        \left(\frac{1}{V}\right)\Bigg|_{V_T}^{V_0}=
        \frac{2}{\xi\Sigma_s}\left(\frac{1}{V_T}-\frac{1}{V_0}\right)
    \end{eqnarray}
    Если в качестве $E_0$ принять среднюю энергию нейтронов деления,
    т.е. \(E_0\approx2\text{Мэв}\), то \(V_0 \gg V_T\)
    и предыдущее выражение еще более упростится
    \begin{equation}\label{e_28}\tag{28}
        t_\text{зам}\cong\frac{2}{\xi\Sigma_s V_T}
    \end{equation}
    В таблице \ref{t_1} представлены значения параметров диффузии и
    замедления для различного вида замедлений
    \begin{table}[b]\centering
        \begin{tabular}{ | l | c | c | c | c | c | }
            \hline
            Замедлитель & Плотность$IO^3$ & $L^2$ & $\tau$ & $t_T$ &
            $t_\text{зам}$ \\
            {} & $\frac{\text{кг}}{\text{м}^2}$ & $\text{см}^2$ &
            $\text{см}^2$ & мс & мкс\\
            \hline
                $H_2 O$ & 1.00 & 7.4 & 27 & 0.21 & 6.7 \\
                $D_2 O$ & 1.10 & 25600 & 120 & 138 & 48 \\
                $Be$ & 1.84 & 441 & 96 & 3.7 & 59 \\
                $BeO$ & 2.96 & 641 & 105 & 6.2 & 76 \\
                $C''$ & 1.60 & 2916 & 350 & 15.2 & 149 \\
            \hline
        \end{tabular}
        \caption{Параметры диффузии и замедления}
        \label{t_1}
    \end{table}
    Из таблицы видно, премя пребывания нейтрона в тепловой области
    примерно на два порядка больше, чем время замедления. Это
    приводит к тому, что число тепловых нейтронов в замедлителе во
    столько же раз больше числа замедляющихся нейтронов, т.е.
    нейтроны "накапливаются" в тепловой области. В ядерных реакторах
    с графитовым замедлителем среднее время жизни нейтрона
    $10^{-3}\text{с}$, а в ядерных реакторах с графитовым
    замедлителем $10^{-4}\text{с}$. В ядерных реакторах на быстрых
    нейтронах, где замедления практически нет, среднее время жизни
    нейтрона $10^{-4}\text{с}$.
    \subsection[Матеметическое моделирование]{Математическое
        моделирование процесса диффузии замедляющихся
        нейтронов от точечного источника в бесконечной непоглащающей
        среде.}

    Входной информацией являются массовые числа ядер, входящих в
    состав рассматриваемой среды и соответствующие макроконстанты
    рассеяния. Например, если среда состоит из углерода, то задается
    \(A=12\), $\Sigma_s^c$; Если же среда состоит из ядер двух
    сортов, например $H_2 O$, то задаются \(A=1\); $\Sigma_S^H$;
    \(B=16\); $\Sigma_S^O$. Задается также энергия нейтронов
    источника $E_0[\text{МэВ}$; Задается координата источника
    нейтронов \(X_M=0\); \(Y_M=0\);\(Z_M=0\).

    \subsection{Алгоритм моделирования}

    1. Разыгрывается длинна свободного пробега нейтрона до
    столкновения с ядром среды
    \begin{equation}\nonumber
        l_1=\frac{1}{\Sigma_{tr}}\ln{\gamma}\text{, где}
    \end{equation}
    $\gamma$ - равномерно распределенная на отрезке $[0,1]$
    случайная величина. Если среда многокомпонентна, то
    \(\Sigma_{tr}=\sum_i^n \Sigma_{tr_i}\), например, для $H_2 O$

    2. Разыгрываются направляющие косинусы движения нейтрона от
    изотропного источника
    \begin{equation}\nonumber
        \omega_z = 1 - 2\gamma;\qquad
        \omega_x = \sqrt{1 - \omega_z^2}\cos{(2\pi\gamma)};\qquad
        \omega_y = \sqrt{1 - \omega_z^2}\sin{(2\pi\gamma)}
    \end{equation}

    3. Рассчитывается точка, где нейтрон столкнулся с ядром:
    \begin{equation}\nonumber
        X_K = X_M + \omega_x l;\qquad
        Y_K = Y_M + \omega_y l;\qquad
        Z_K = Z_M + \omega_z l;\qquad
        r_K = \sqrt{X_K^2 + Y_K^2 + Z_K^2}
    \end{equation}
    Если среда двухкомпонентная, то определяется с какого сорта ядром
    столкнулся нейтрон. Для этого разыгрывается $\gamma$ из $[0,1]$ и
    если $\gamma < \frac{\Sigma_S1}{\Sigma_S}$, то нейтрон столкнулся
    с ядром под условным номером $1$, если
    $\gamma > \frac{\Sigma_S1}{\Sigma_S}$, то нейтрон столкнулся с
    ядром под условным номером $2$. Например: для $H_2 O$
    макросечение рассеяния состоит из двух слогаемых
    \begin{equation}\nonumber
        \Sigma_s^{H_2 O} =
        \underbrace{2\partial_S^H N_H}_{\Sigma_{s1}}
        \underbrace{\partial_S^{O_2}N^{O_2}}_{\Sigma_{s2}}
    \end{equation}
    Тогда, если $\gamma<\frac{\Sigma_S1}{\Sigma_s^{H_2 O}}$, то
    нейтрон столкнулся с ядром водорода, в противном случае
    $\gamma<\frac{\Sigma_S1}{\Sigma_s^{H_2 O}}$ - нейтрон столкнулся
    с ядром кислорода.

    4. После того, как определен атомный номер ядра, с которым
    столкнулся нейтрон (пусть этот номер А), разыгрывается случайная
    величина $\gamma$ из интервала $[0,1]$
    \begin{eqnarray}\nonumber
        1. \cos{Q} = 1 - 2 \gamma = \omega_z
        \omega_x = \sqrt{1-\omega_z^2}\cos{(2\pi\gamma)}
        \omega_y = \sqrt{1-\omega_z^2}\sin{(2\pi\gamma)}
        l_s = - \frac{ln{\gamma}}{\Sigma_{tr}}
        \\ \nonumber
        2. \epsilon = \frac{{(A-1)}^2}{{(A+1)}^2}
        \\ \nonumber
        3. E_1 = \frac{E_0}{2}[(1 + \epsilon) + (1 - \epsilon)
        \cos{\theta}]
    \end{eqnarray}
    Таким образом, определятся энергия нейтрона после столкновения
    $E_1$, направляющие косинусы движения нейтрона после рассеяния
    $\omega_x$, $\omega_y$, $\omega_z$, а значит и координаты
    следующего столкновения.

    5. Если $E_1 \ge E_T$, то возврат к п.4 в котором следует
    положить $E_0=E_1$, если $E < E_T$, то разыгрывается новый
    нейтрон источника, т.е. программа должна идти на п.1.

    \subsection{Выходная информация}
    Выходной информацией являются для каждого из $M$
    рассмотренных нейтронов источника следующиемассивы: координаты
    точек столкновения нейтрона с ядрами среды и энергия нейтрона
    после столкновения. В результате обработки этих массивов
    информации можно получить экспериментальные значения возраста
    нейтронов в зависимости от энергии и распределения плотности
    замедления. Действительно, так как возраст нейтронов энергии
    $E$ связан со средним квадратом смещения нейтрона соотношением
    \(\tau(E)=\frac{1}{\Omega}\vec{r}^2(E)\), то достаточно
    определить средний квадрат смещения нейтронов от источника до
    точки замедления до энергии $E$. Зафиксируем некоторое заданное
    значение энергии нейтрона $E$ и для каждого из $M$ рассмотренных
    нейтронов источника определим координаты точки рассеяния, в
    результате которого энергия нейтрона станет меньше, чем $E$.
    Пусть координаты этой точки для $i$-ого нейтрона будут
    $(x_i, y_i, z_i)$. Тогда средний квадрат смещения нейтрона до
    замедления его до энергии $E$ будет приблеженно определяться
    выражением
    \begin{equation}\nonumber
        \bar{r}^2 = \frac{1}{M}\sum_{i=1}^{M}
        (x_i^2 + y_i^2 + z_i^2)
    \end{equation}
    Задавая различные значения величины $E$, можно получить
    зависимость возраста от энергии
    \begin{equation}\nonumber
        \tau(E)=\frac{1}{6}\bar{r}^2(E)
    \end{equation}
    При $E=E_T$ возраст нейтрона $\tau$ характеризует смещение
    нейтрона от точки его рождения до точки превращения замедляющегося
    нейтрона в тепловой.

    Экспериментальное распределение плотности замедления по
    пространству при различных значениях возраста нейтронов можно
    получить следующим образом. Зададимся обастью изменения
    координаты $r$ в сферической геометрии
    \(r = \sqrt{x_2 + y_2 + z^2)}\). Пусть эта величина будет равна
    $R$. Разобьем радиус вектор $R$ на $K$ частей, тогда
    \(R_i=i\frac{R}{K} = i_\Delta R\). Введем в рассмотрение объем
    пространства, заключенного между двумя соседними сферами
    \begin{equation}\nonumber
        V_i = \frac{4}{3}\pi(R_{i+1}^3 - R_i^3), i=0,\dots,k
    \end{equation}
    Зададимся величиной энергии $E$ и номером $i$, используя
    информацию о координатах столкновения нейтронов с ядрами среды
    и об их энергии, определим, как и прежде, координаты точки
    рассеяния, в результате которого энергия нейтрона станет меньше,
    чем $E$. Пусть эта точка характеризуется радиусом
    \(r=\sqrt{x^2+y^2+z^2}\). Определим радиусы этих точек для всех
    $M$, рассмотренных нейтронов источника. Далее вычислим
    относительную долю нейтронов из $M$ рассмотренных координаты
    которых, попали в пространство $V_i$, т.е. определим величину
    \begin{equation}\nonumber
        \tilde{q}(R_i)=\frac{n_i}{M}\frac{1}{V_i}
    \end{equation}
    где $n_i(E)$ - число нейтронов из $M$ рассмотренных, величина
    радиуса смещения которых оказалась в пределах объема
    Величина $\tilde{q}$ будет пропорциональна плотности замедления.

    \subsection{Подготовка к данному разделу лабороторной работы}
    1. Изучить теоретический материал.
    \newline
    2. Для заданного вариантом состава среды рассчитать $\Sigma_{tr}$,
    $D$, $\xi$. В заданном энергетическом диапозоне $E_0 \div E_T$,
    построить зависимость $\tau(E)$.
    \newline
    3. Построить для заданных свойств зависимости $q(\vec{r},E)$ и
    $q(\vec{r},\tau)$ в случае точечного источника в бесконечной
    непоглощающей среде.
    \newline
    4. Для заданного варианта состава среды определить время
    замедления от энергии $E_0$ до тепловой энергии и время диффузии.
    \newline
    5. Нарисовать блок схему алгоритма модели.
    \newline
    6. Разработать план исследования процесса замедления при диффузии.

    \subsection{Подготовка к сдаче данного раздела лабораторной работы}
    1. По данным распечаткам построить для двух из рассмотренных $M$
    судеб нейтронов зависимость $E(r^2)$, где \(r^2 = x^2 + y^2 + z^2\)
    \newline
    2. Получить зависимость экспериментального значения возраста
    нейтронов от числа рассмотренных судеб нейтронов. Сравнить с
    теоретическими значениями возраста.
    \newline
    3. Построить зависимость возраста от энергии по данным численного
    эксперимента.
    \newline
    4. Построить экспериментальную зависимость $\tilde{q}(\vec{r},\tau)$
    и сравнить с аналитической зависимостью. 

    Основные результаты, полученные при выполнении всех разделов
    лабороторной работы излагаются в заключении.

    При сдаче лабораторной работы необходимо правильно отвечать на
    следующие \underline{контрольные вопросы}:

    1. На основании каких физических законов сохраниения, получены
    выражения для описания акта рассеяния нейтрона на ядре?
    \newline
    2. Какой элемент эффективнее всего замедляет нейтроны?
    \newline
    3. Какая величина используется для характеристики качества
    замедлителя?
    \newline
    4. В какой системе координат рассеяние нейтронов практически
    считается сфериеским - симметричным?
    \newline
    5. Какой физический смысл $\Sigma_{tr}$?
    \newline
    6. Что такое летаргия нейтрона? Как с помощью летаргии определить
    среднее число столкновений нейтронов?
    \newline
    7. От чего зависит средняя логарифмическая потеря энергии
    нейтронов при замедлении?
    \newline
    8. Из каких соображений можно получить спектр замедляющихся
    нейтронов в поглащающей среде?
    \newline
    9. Что такое "спектр Ферми"? Почему с уменьшением энергии
    нейтронов $\varphi(E)$ растет?
    \newline
    10. Понятие плотности замедления. Как связана плотность замедления
    со спектром нейтронов?
    \newline
    11. Какие основные параметры процесса диффузии нейтронов и какой
    их физический смысл?
    \newline
    12. Смысл величины $L^2$?
    \newline
    13. Какова качественная зависимость $L^2$ от температуры среды?
    \newline
    14. Смысл членов уравнения диффузии нейтронов в среде. Как они
    зависят от параметров среды?
    \newline
    15. Условия однозначности для уравнения диффузии.
    \newline
    16. Уравнение возраста. Смысл членов уравнения.
    \newline
    17. Понятие возраста нейтронов в среде. От каких физических
    свойств среды зависит величина возраста нейтронов?
    \newline
    18. Что такое площадь миграции?
    \newline
    19. Каковы характерные величины време диффузии и замедления в
    воде, тяжелой воде и графите?

    \subsection{Литература}
    1. Лекции по курсу "Математические модели физических систем".
    \newline
    2. Климов А.Н. Ядерная физика и ядерные реакторы, М.,
    Энергоатомиздат, 1985, с. 167-195.
    \newline
    3. Бартоломей Г.Г. и др. Основы теории и методы расчета ядерных
    энергетических реакторов. М., Энергоиздат, 1982, с. 81-159
\end{document}
