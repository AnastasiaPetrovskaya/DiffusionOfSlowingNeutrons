\documentclass[a4paper]{article}

\usepackage[english, russian]{babel}
\usepackage[utf8x]{inputenc}
\usepackage{amsmath}
\usepackage{graphicx}
\usepackage[colorinlistoftodos]{todonotes}
\usepackage[left=2.5cm, top=2cm, right=1.5cm, bottom=2.5cm,
            nohead]{geometry}

\title{Диффузия замедляющихся нейтронов.}
\author{
    K01-22М:
    Петровская А.В., Бодунков Д.В.,
    Таракчян Л.С., Минаков А.О.}

\begin{document}
    \maketitle

    В предыдущих разделах рассмотрены процессы замедления
    и диффузии в отрыве друг от друга. При рассмотрении процесса
    замедления не учитывался факт пространственного перемещения,
    а при изучении диффузии нейтронов не учитывался их факт
    пространственного перемещения, а при изучении диффузии нейтронов
    пренебрегалось изменениями энергии нейтронов при рассеянии на
    ядрах среды. В действительности эти процессы происходят
    одновременно: нейтроны сталкиваясь с ядрами среды перемещаются
    в пространстве и изменяют свою энергию. Поэтому при вычислении
    распределения плотности потока нейтронов в ядерном реакторе
    нельзя разделять процессы замедления и диффузии. Наиболее
    простой математической моделью, позволяющей описать диффузию
    замедляющихся нейтронов является модель непрерывного замедления.
    \newline
    Основное положение этой модели заключаетмя в том, что
    дискретный процесс потери энергии нейтроном, при замедлении
    аппроскимируется непрерывной зависимостью ( см. рис )
    \newline
    Найдем функциональную связь между временем и энергией
    при непрерывном торможении нейтрона. Пусть нейтрон при
    своем замедлени проходит энергетический интервал $dE$,
    около энергии $E$ за время $dt$. Нейтрон снижает свою
    энергию за счет того, что за время dt сталкивается с ядрами среды.
    \newline
    Число таких столкновений при диффузии нейтрона легко определяется 
    из соотношений
    \begin{equation}
        \frac{V}{l_s}dt
        \text{, где $V$ - скорость нейтрона}
    \end{equation}
    соответствующая энергия $E$.
    \newline
    С другой стороны, число столкновений, которое необходимо
    претерпеть нейтрону, чтобы изменить свою энергию на величину $dE$,
    есть отношение приращения логарифма энергии на этом интервале к
    величине $\xi$ - средней потере логарифма энергии на одно
    столкновение. Приравняем эти величины и ваполняя простые
    преобразования, получим:
    \begin{equation}
        \frac{dE}{dt} = -\frac{V}{l_s}\xi E = \xi \Sigma_s V E
    \end{equation}
    Знак ($-$) в этом выражении взят с целью описать факт уменьшения
    энергии нейтрона со временем.
    \newline
    Обратимся теперь к следующей задаче: в бесконечной непоглощающей
    среде находится точечный источник, испускающий нейтроны с энергией
    $E_o$. Если источник испускает в единицу времени какую-то порцию
    нейтронов, то эти нейтроны будут распределяться по все
    возрастающему объему. Поэтому число нетронов в $I \text{ см}^3$
    около точки с координатой $\overrightarrow{r}$, будет зависеть от
    хронологического времени
    $t$, т.е. \( n_1 = n_1 (\overrightarrow{r},t) \).
    \newline
    Изменение плотности нейтронов $n_1(\overrightarrow{r},t)$ при
    отсутствии поглощения происходит только за счет диффузии, поэтому:
    \begin{equation}
        \frac{\partial n_1}{\partial t} = D V_\Delta n_1
    \end{equation}
    \newline
    Уравнение (3) описывает изменение плотности нейтронов, за счет
    того, что источник испустил порцию нейтронов, равную мощности
    источники, то есть, по сути дела, уравнения (3) описывает скорость
    изменения числа нейтронов, т.е. 
    \(n_1(\overrightarrow{r}, t)=\frac{dn}{dt}\).
    Учтем, что переменные $t$ и $E$ связанны соотношением (1).
    Поскольку форма дифференциала $dn$ не зависит от того, что
    рассматривать в качемтве переменной, имеем
    \begin{equation}
        dn = \frac{dn}{dt}dt \text{\;\;\;\;\;\;или\;\;\;\;\;\;}
        dn = \frac{dn}{dE}dE
    \end{equation}
    откуда
    \begin{equation}
        \frac{dn}{dt}dt = \frac{dn}{dE}dE
    \end{equation}
    Обозначим \(\frac{dn}{dE}=n_2(\overrightarrow{r},E)\), тогда
    будем иметь
    \begin{equation}
        n_1(\overrightarrow{r},t)dt = n_2(\overrightarrow{r},E)dE
    \end{equation}
    откуда
    \begin{equation}
        n_1(\overrightarrow{r},t) =
        n_2(\overrightarrow{r},E)\frac{dE}{dt} =
        n_2(\overrightarrow{r},E) V \xi E \Sigma_s
    \end{equation}
    или
    \begin{equation}
        n_1(\overrightarrow{r},t) =
        \varPhi(\overrightarrow{r}, E) \xi \Sigma_s E
    \end{equation}
    \newline
    $\frac{dn}{dE}$ - есть число нейтронов в ед. объема,
    приходящихся на единичный энергетический интервал, т.е.
    $\varphi(\overrightarrow{r},E)$.
    \newline
    Величина \(q(\overrightarrow{r},\overrightarrow{E})=
    \xi\Sigma_s E\varphi(\overrightarrow{r},E) \)
    носит название плотности замедления и имеет смысл числа нейтронов
    в $I\text{ см}^3$ пересекающих в ед. времени значение энергии $E$.
    \newline
    Действительно, величина $\xi$ есть среднее изменение логорифма
    энергии в одном акте рассеяния
    \begin{equation}
        \xi = \overline{_\Delta ln E} \approx
        \overline{(ln E)^{'}_E {}_\Delta E} =
        \overline{\frac{1}{E} {}_\Delta E}
        \text{, откуда } \overline{{}_\Delta E} = \xi E
    \end{equation}
    \newline
    $\overline{{}_\Delta E}$ - потеря энергии нейтроном в одном акте
    рассеяния. Если интервал $\overline{{}_\Delta E}$ расположен
    между $E$ и $E + \overline{{}_\Delta E}$, то каждое рассеяние
    приводит к снижению энергии нейтрона за значение $E$.
    \newline
    Число нейтронов претерпевших рассеяние в интервале
    $[E, E + \overline{{}_\Delta E}]$, есть произведение числа
    нейтронов рассеяных в единичном интервале энергий
    $\varphi(E)\Sigma_s$ на величину
    $\overline{{}_\Delta E}$. Все эти нейтроны снижают свою энергию
    за значение $E$, следовательно
    \begin{equation}
        q(\overrightarrow{r},E) =
        \varphi(E)\Sigma_s\overline{{}_\Delta E} =
        \Sigma_s\varphi(E)E
    \end{equation}
    \newline
    Так как \(\frac{\partial n_1}{\partial E} =
    \frac{\partial n_1}{\partial E} \frac{\partial E}{\partial t}\)
    , и \(n_1 = q(\overrightarrow{r},E)\), получим из уравнения (3)
    относительно плотности нейтронов, уравнение (23) относительно
    плотности замедления
    \begin{equation}
        D_\Delta q(\overrightarrow{r}, E) = -\xi E\Sigma_s
        \frac{\partial q(\overrightarrow{r}, E)}{\partial E}
    \end{equation}
    Уравнение (23) можно еще упростить, если ввести новую независимую
    переменную
    \begin{equation}\nonumber
        \tau(E) = \int_E^{E_0} \frac{D}{\Sigma_s}\frac{dE}{\xi E}
    \end{equation}
    \begin{equation}
        \text{Очевидно, что }
        \frac{\partial q}{\partial \tau}
        \frac{\partial \tau}{\partial E} =
        \frac{\partial q}{\partial E}
        \text{; откуда}
    \end{equation}
    \begin{equation}
        \frac{\partial q}{\partial \tau} =
        \frac{\partial q}{\partial E}
        \frac{1}{\frac{\partial \tau}{\partial E}}
        \text{ , но }
        \frac{\partial \tau}{\partial E} =
        -\frac{D}{\Sigma_s\xi E}
    \end{equation}
    \begin{equation}\nonumber
        \text{т.е. }
        \frac{\partial q}{\partial \tau} =
        -\frac{\xi\Sigma_s E}{D}\frac{\partial q}{\partial E}
    \end{equation}
    Тогда уравнение (23) запишется в следующем виде
    \begin{equation}\label{e_24}\tag{24}
        {}_\Delta q(\overrightarrow{r}, \tau) =
        \frac{\partial q(\overrightarrow{r}, \tau)}{\partial \tau}
    \end{equation}
    Уравнение (\ref{e_24}) описывает распределение в пространстве
    $\overrightarrow{r}$ и в пространстве $\tau$ плотности
    замедления $q(\overrightarrow{r}, \tau)$. Величина $\tau$ - носит
    специальное название - "возраст нейтрона" и имеет размерность
    $[\text{см}^2]$. Само уравнение (\ref{e_24}) является уравнением
    теплопроводности. Решение этого уравнения для точечного источника
    в бесконечной среде имеет вид:
    \begin{equation}\label{e_25}\tag{25}
        q(\overrightarrow{r},\tau) =
        \frac{Q}{(4\pi\tau)^{3/2}}
        \exp{(-\frac{r^3}{4\tau})}
    \end{equation}
    На рис. 9 показан качественный вид решения (\ref{e_25}) в
    зависимости от координаты при различных значениях параметра $\tau$.
    Если $\tau$ мало, то это означает, что энергия нейтронов
    достаточно близка к энергии нейтронов источника $E_0$ и кривая
    $q(r,\tau)$ становится более выровненным. Важным случаем является
    тот, когда
    \begin{equation}\nonumber
        \tau=\int_{E_t}^{E_0}\frac{D}{\xi\Sigma_s}\frac{dE}{E}=
        \tau_\tau
    \end{equation}
    где $E_T$ - энергия тепловых нейтронов. В этом случае 
    $q(r, \tau_\tau)$ - дает распределение источников тепловых
    нейтронов около точечного источника быстрых нейтронов. Физический
    смысл понятия возраста нейтронов $\tau$ заключается в том, что
    возраст нейтронов $\tau(E)$ есть величина пропорциональная среднему
    квадрату смещения нейтронов от точки их рождения до точки, где
    их энергия равна величине $E$. Действительно, средний квадрат
    смещения нейтрона до достижения возраста $\tau$ есть
    \begin{equation}
        \overrightarrow{r}_{\tau}^2 =
        frac{\int_0^{\infty}r^2 q(r, \tau)4\pi r^2 dr}
        {\int_0^{\infty}q(r,\tau)\pi r^2 dr} =
        \frac{1}{Q}\frac{Q4\pi}{{4\pi\tau}^{3/2}}
        \int_0^\infty r^4 e^{-\frac{r^2}{t\tau}}dr = 6\tau
    \end{equation}
    При получении этого результата \(\overrightarrow{r}^2 = 6\tau\)
    учтено, что \(\int_0^\infty q(r,\tau)4\pi r^2 dr\), т.е. число
    нейтронов замедляющихся до возраста $\tau$ в ед. времени во всем
    объеме рассматриваемой среды равно мощности источника $Q$.
    Поскольку возраст нейтронов $\tau(E_T)$ пропорционален смещению
    нейтрона от точки рождения (в качестве быстрого нейтрона) до
    точки замедления до тепловой энергии $E_T$, а квадрат длины
    диффузии $L^2$ пропорционален смещению от точки рождения
    теплового нейтрона до точки поглощения, то величина 
    \(M^2 = \tau + L^2\) - пропорциональна среднему смещению нейтрона
    от точки его рождения как быстрого нейтрона до точки его
    поглощения как теплового нейтрона ( см.рис. 10). Величина
    $M^2$ называется площадью миграции нейтрона.
    \newline
    1 - точка, где родился быстрый нейтрон
    \newline
    2 - точка, где быстрый нейтрон, замедлился до тепловой
    энергии и стал тепловым (точка рождения теплового нейтрона)
    \newline
    3 - точка поглощения теплового нейтрона
    \newline
    Важными характеристиками являются время диффузии и время
    замедления нейтронов до тепловой энергии. При нормальных условаиях
    в качестве тепловой энергии нейтрона $E_T$ принимается величина
    \(E_T=0,025\text{эВ}\), что соответствует скорости нейтронов
    \(V_T=2200\frac{\text{м}}{\text{сек}}\).
    \newline
    Среднее время диффузии нейтрона до поглощения определяется из
    выражения
    \begin{equation}\label{e_27}\tag{27}
        t_T=\frac{l_U}{V_T}=\frac{1}{\Sigma_u V_T}
    \end{equation}
    $\Sigma_u$ - макроскопическое сечение поглощения среды
    \newline
    Среднее время замедления нейтрона от энергии $E_0$ до энергии
    $E_T$ определяется с помощью выражения (20)
    \begin{equation}\nonumber
        t_\text{зам} = \int_0^{t_\text{зам}} dt =
        \int_{E_T}^{E_0} - \frac{dE}{\xi \Sigma_s VE} =
        \int_{E_T}^{E_0} - \frac{dE}{\xi \Sigma TODO E^{3/2}} =
        \frac{1}{TODO \xi \Sigma_s \oint}
    \end{equation}
    Если в качестве $E_0$ принять среднюю энергию нейтронов деления,
    т.е. \(E_0 TODO{приблизительно} 2\text{Мэв}\), то \(V_0 TODO{стремится} V_T\)
    и предыдущее выражение еще более упростится
    \begin{equation}\label{e_28}\tag{28}
        t_\text{зам} TODO \frac{2}{\xi\Sigma_s V_T}
    \end{equation}
    В таблице TODO представлены значения параметров диффузии и
    замедления для различного вида замедлений
    TODO Таблица1
    Из таблицы видно, премя пребывания нейтрона в тепловой области
    примерно на два порядка больше, чем время замедления. Это
    приводит к тому, что число тепловых нейтронов в замедлителе во
    столько же раз больше числа замедляющихся нейтронов, т.е.
    нейтроны "накапливаются" в тепловой области. В ядерных реакторах
    с графитовым замедлителем среднее время жизни нейтрона
    $10^{-3}\text{с}$, а в ядерных реакторах с графитовым
    замедлителем $10^{-4}\text{с}$. В ядерных реакторах на быстрых
    нейтронах, где замедления практически нет, среднее время жизни
    нейтрона $10^{-4}\text{с}$.
    TODO Математическое моделирование процесса диффузии замедляющихся
    нейтронов от точечного источника в бесконечной непоглащающей
    среде.
    \newline
    Входной информацией являются массовые числа ядер, входящих в
    состав рассматриваемой среды и соответствующие макроконстанты
    рассеяния. Например, если среда состоит из углерода, то задается
    \(A=12\), $\Sigma_s^c$; Если же среда состоит из ядер двух
    сортов, например $H_2 O$, то задаются \(A=1\); $\Sigma_S^H$;
    \(B=16\); $Sigma_S^O$. Задается также энергия нейтронов
    источника $E_0[\text{МэВ}$; Задается координата источника
    нейтронов \(X_M=0\); \(Y_M=0\);\(Z_M=0\).
    TODO Алгоритм моделирования
    1. Разыгрывается длинна свободного пробега нейтрона до
    столкновения с ядром среды
    \begin{equation}\nonumber
        l_1=\frac{1}{\Sigma_{tr}}\ln{\gamma}\text{, где}
    \end{equation}
    $\gamma$ - равномерно распределенная на отрезке $[0,1]$
    случайная величина. Если среда многокомпонентна, то
    \(\Sigma_{tr}=\sum_i^n \Sigma_{tr_i}\), например, для $H_2 O$
    \newline
    2. Разыгрываются направляющие косинусы движения нейтрона от
    изотропного источника
    \begin{equation}\nonumber
        \omega_z = 1 - 2\gamma;\qquad
        \omega_x = TODOSQRT{1 - \omega_z^2}\cos{(2\pi\gamma)};\qquad
        \omega_y = TODOSQRT{1 - \omega_z^2}\sin{(2\pi\gamma)}
    \end{equation}
    \newline
    3. Рассчитывается точка, где нейтрон столкнулся с ядром:
    \begin{equation}\nonumber
        X_K = X_M + \omega_x l;\qquad
        Y_K = Y_M + \omega_y l;\qquad
        Z_K = Z_M + \omega_z l;\qquad
        r_K = TODOSQRT{X_K^2 + Y_K^2 + Z_K^2}
    \end{equation}
    Если среда двухкомпонентная, то определяется с какого сорта ядром
    столкнулся нейтрон. Для этого разыгрывается $\gamma$ из $[0,1]$ и
    если $\gamma < \frac{\Sigma_S1}{\Sigma_S}$, то нейтрон столкнулся
    с ядром под условным номером $1$, если
    $\gamma > \frac{\Sigma_S1}{\Sigma_S}$, то нейтрон столкнулся с
    ядром под условным номером $2$. Например: для $H_2 O$
    макросечение рассеяния состоит из двух слогаемых
    \begin{equation}\nonumber
        \Sigma_s^{H_2 O} = 
    \end{equation}


\end{document}
