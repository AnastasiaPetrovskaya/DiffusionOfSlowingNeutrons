\documentclass[a4paper]{article}

\usepackage[english, russian]{babel}
\usepackage[utf8x]{inputenc}
\usepackage{amsmath}
\usepackage{graphicx}
\usepackage[colorinlistoftodos]{todonotes}

\title{Диффузия замедляющихся нейтроноДиффузия замедляющихся нейтронов.}
\author{K01-22М: Петровская, Бодунков, Таракчян, Царь}

\begin{document}
\maketitle
dkfjhksdjлваорыолвфра
воларловр
\begin{abstract}
Your abstract.
\end{abstract}

\section{Introduction}
    Найдем функциональную связь между временем и энергией
    при непрерывном торможении нейтрона. Пусть нейтрон при
    своем замедлени проходит энергетический интервал $dE$,
    около энергии $E$ за время $dt$. Нейтрон снижает свою
    энергию за счет того, что за время dt сталкивается с ядрами среды.
    \newline
    Число таких столкновений при диффузии нейтрона легко определяется 
    из соотношений
    \begin{equation}
        \frac{V}{l_s}dt
        \text{, где $V$ - скорость нейтрона}
    \end{equation}
    соответствующая энергия $E$.
    \newline
    С другой стороны, число столкновений, которое необходимо
    претерпеть нейтрону, чтобы изменить свою энергию на величину $dE$,
    есть отношение приращения логарифма энергии на этом интервале к
    величине $\xi$ - средней потере логарифма энергии на одно
    столкновение. Приравняем эти величины и ваполняя простые
    преобразования, получим:
    \begin{equation}
        \frac{dE}{dt} = -\frac{V}{l_s}\xi E = \xi \Sigma_s V E
    \end{equation}
    Знак ($-$) в этом выражении взят с целью описать факт уменьшения
    энергии нейтрона со временем.
    \newline
    Обратимся теперь к следующей задаче: в бесконечной непоглощающей
    среде находится точечный источник, испускающий нейтроны с энергией
    $E_o$. Если источник испускает в единицу времени какую-то порцию
    нейтронов, то эти нейтроны будут распределяться по все
    возрастающему объему. Поэтому число нетронов в $I \text{ см}^3$
    около точки с координатой $\overrightarrow{r}$, будет зависеть от
    хронологического времени
    $t$, т.е. \( n_1 = n_1 (\overrightarrow{r},t) \).
    \newline
    Изменение плотности нейтронов $n_1(\overrightarrow{r},t)$ при
    отсутствии поглощения происходит только за счет диффузии, поэтому:
    \begin{equation}
        \frac{\partial n_1}{\partial t} = D V_\Delta n_1
    \end{equation}
    \newline
    Уравнение (3) описывает изменение плотности нейтронов, за счет
    того, что источник испустил порцию нейтронов, равную мощности
    источники, то есть, по сути дела, уравнения (3) описывает скорость
    изменения числа нейтронов, т.е. 
    \(n_1(\overrightarrow{r}, t)=\frac{dn}{dt}\).
    Учтем, что переменные $t$ и $E$ связанны соотношением (1).
    Поскольку форма дифференциала $dn$ не зависит от того, что
    рассматривать в качемтве переменной, имеем
    \begin{equation}
        dn = \frac{dn}{dt}dt \text{\;\;\;\;\;\;или\;\;\;\;\;\;}
        dn = \frac{dn}{dE}dE
    \end{equation}
    откуда
    \begin{equation}
        \frac{dn}{dt}dt = \frac{dn}{dE}dE
    \end{equation}
    Обозначим \(\frac{dn}{dE}=n_2(\overrightarrow{r},E)\), тогда
    будем иметь
    \begin{equation}
        n_1(\overrightarrow{r},t)dt = n_2(\overrightarrow{r},E)dE
    \end{equation}
    откуда
    \begin{equation}
        n_1(\overrightarrow{r},t) =
        n_2(\overrightarrow{r},E)\frac{dE}{dt} =
        n_2(\overrightarrow{r},E) V \xi E \Sigma_s
    \end{equation}
    или
    \begin{equation}
        n_1(\overrightarrow{r},t) =
        \varPhi(\overrightarrow{r}, E) \xi \Sigma_s E
    \end{equation}
    \newline
    $\frac{dn}{dE}$ - есть число нейтронов в ед. объема,
    приходящихся на единичный энергетический интервал, т.е.
    $\varphi(\overrightarrow{r},E)$.
    \newline
    Величина \(q(\overrightarrow{r},\overrightarrow{E})=
    \xi\Sigma_s E\varphi(\overrightarrow{r},E) \)
    носит название плотности замедления и имеет смысл числа нейтронов
    в $I\text{ см}^3$ пересекающих в ед. времени значение энергии $E$.
    \newline
    Действительно, величина $\xi$ есть среднее изменение логорифма
    энергии в одном акте рассеяния
    \begin{equation}
        \xi = \overline{_\Delta ln E} \approx
        \overline{(ln E)^{'}_E {}_\Delta E} =
        \overline{\frac{1}{E} {}_\Delta E}
        \text{, откуда } \overline{{}_\Delta E} = \xi E
    \end{equation}
    \newline
    $\overline{{}_\Delta E}$ - потеря энергии нейтроном в одном акте
    рассеяния. Если интервал $\overline{{}_\Delta E}$ расположен
    между $E$ и $E + \overline{{}_\Delta E}$, то каждое рассеяние
    приводит к снижению энергии нейтрона за значение $E$.
    \newline
    Число нейтронов претерпевших рассеяние в интервале
    $[E, E + \overline{{}_\Delta E}]$, есть произведение числа
    нейтронов рассеяных в единичном интервале энергий
    $\varphi(E)\Sigma_s$ на величину
    $\overline{{}_\Delta E}$. Все эти нейтроны снижают свою энергию
    за значение $E$, следовательно
    \begin{equation}
        q(\overrightarrow{r},E) =
        \varphi(E)\Sigma_s\overline{{}_\Delta E} =
        \Sigma_s\varphi(E)E
    \end{equation}
    \newline
    Так как \(\frac{\partial n_1}{\partial E} =
    \frac{\partial n_1}{\partial E} \frac{\partial E}{\partial t}\)
    , и \(n_1 = q(\overrightarrow{r},E)\), получим из уравнения (3)
    относительно плотности нейтронов, уравнение (23) относительно
    плотности замедления
    \begin{equation}
        D_\Delta q(\overrightarrow{r}, E) = -\xi E\Sigma_s
        \frac{\partial q(\overrightarrow{r}, E)}{\partial E}
    \end{equation}
    Уравнение (23) можно еще упростить, если ввести новую независимую
    переменную
    \begin{equation}
        \tau(E) = \int_E^{E_0} \frac{D}{\Sigma_s}\frac{dE}{\xi E}
    \end{equation}
    \begin{equation}
        \text{Очевидно, что }
        \frac{\partial q}{\partial \tau}
        \frac{\partial \tau}{\partial E} =
        \frac{\partial q}{\partial E}
        \text{; откуда}
    \end{equation}
    \begin{equation}
        \frac{\partial q}{\partial \tau} =
        \frac{\partial q}{\partial E}
        \frac{1}{\frac{\partial \tau}{\partial E}}
        \text{ , но }
        \frac{\partial \tau}{\partial E} =
        -\frac{D}{\Sigma_s\xi E}
    \end{equation}
    \begin{equation}
        \text{т.е. }
        \frac{\partial q}{\partial \tau} =
        -\frac{\xi\Sigma_s E}{D}\frac{\partial q}{\partial E}
    \end{equation}
    Тогда уравнение (23) запишется в следующем виде
    \begin{equation}\label{eq_24}\tag{24}
        {}_\Delta q(\overrightarrow{r}, \tau) =
        \frac{\partial q(\overrightarrow{r}, \tau)}{\partial \tau}
    \end{equation}
    Уравнение (\ref{eq_24}) описывает распределение в пространстве
    $\overrightarrow{r}$ и в пространстве $\tau$ плотности
    замедления $q(\overrightarrow{r}, \tau)$.
\end{document}
